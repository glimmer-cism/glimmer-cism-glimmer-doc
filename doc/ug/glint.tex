\newpage
\subsection{GLINT driver}
\subsubsection{Overview}
%
GLINT is the most complex of the drivers supplied as part of GLIMMER. It was
originally developed as an interface between GLIDE and the GENIE Earth-system
model, but is designed to be flexible enough to be used with a wide range of
global climate models. Perhaps the most distinctive feature of GLINT is the
way it uses the object-oriented GLIDE architecture to enable multiple ice
models to be coupled to the same climate model. The means that regional ice
models can be run at high resolution over several parts of the globe, but
without the expense of running a global ice model.

GLINT automates the processes required in coupling regional models to a global
model, particularly the down- and up-scaling of the fields that form the
interface between the two models. The user may specify map projection
parameters for each of the ice models (known as \emph{instances}), and choose
one of several alternative mass-balance schemes to use in the coupling. The
differing time-steps of global model, mass-balance scheme, and ice model are
handled automatically by temporal averaging or accumulation of quantities (as
appropriate). This is illustrated schematically in figure~\ref{ug.fig.glint_timesteps}.  
%
\begin{figure}[htbp]
  \centering
  \includegraphics[width=0.6\textwidth]{\dir/figs/glint_timesteps.eps}
  \caption{Relationship between the timesteps in GLINT. The filled circles
  represent timesteps, the rectangles represent averaging/accumulation, and the arrows,
  flow of coupling fields.}
  \label{ug.fig.glint_timesteps}
\end{figure}
%
\subsubsection{Prerequisites}
%
In order to use GLIMMER, the following should be borne in mind:
%
\begin{itemize}
\item Global input fields must be supplied on a latitude-longitude
  grid. The grid does not have to be uniform in latitude, meaning that
  Gaussian grids may be used. Irregular grids (e.g. icosahedral grids) are not
  supported currently. The boundaries of the grid boxes may be specified; if
  not, they are assumed to lie half-way between the grid-points in lat-lon space.
\item In the global field arrays, latitude must be indexed from north to south
  -- i.e. the first row of the array is the northern-most one. Again, some
  flexibility might be introduced into this in the future.
\item The global grid must not have grid points at either of the
  poles. This restriction is not expected to be permanent, but in the meantime
  can probably be overcome by moving the location of the polar points to be
  fractionally short of the pole (e.g. at 89.9$^{\circ}$ and -89.9$^{\circ}$).
\end{itemize}
%
\subsubsection{Initialising and calling}

The easiest way to learn how GLINT is used is by way of an
example. GLINT should be built automatically as part of GLIMMER, and we assume
here that this has been achieved successfully.

Typically, GLINT will be called from the main program body of a
climate model. To make this possible, the compiler needs to be told to use the
GLINT code. Use statements appear at the very beginning of f90 program
units, before even \texttt{implicit none}:
%
\begin{verbatim}
  use glint_main
\end{verbatim}
%
The next task is to declare a variable of type \texttt{glint\_params}, which
holds everything relating to the model, including any number of ice-sheet
instances:
%
\begin{verbatim}
  type(glint_params) :: ice_sheet
\end{verbatim}
%
Before the ice-sheet model may be called from the climate model, it must be
initialised. This is done with the following subroutine call:
%
\begin{verbatim}
  call initialise_glint(ice_sheet,lats,lons,paramfile)
\end{verbatim}
%
In this call, the arguments are as follows:
%
\begin{itemize}
\item \texttt{ice\_sheet} is the variable of type \texttt{glint\_params}
 defined above;
\item \texttt{lats} and \texttt{lons} are one-dimensional arrays giving the
  locations of the global grid-points in latitude and longitude, respectively; 
\item \texttt{paramfile} is the name of the GLINT configuration file.
\end{itemize}
%
The contents of the namelist files will be dealt with later. Having
initialised the model, it may now be called as part of the main climate
model time-step loop:
%
\begin{verbatim}
    call glint(ice_sheet,time,temp,precip,zonwind,merwind,orog)
\end{verbatim} 
%
The arguments given in this example are the compulsory ones only; a number of
optional arguments may be specified -- these are detailed in the reference
section below. The compulsory arguments are:
%
\begin{itemize}
\item \texttt{ice\_sheet} is the variable of type \texttt{glint\_params}
 defined above;
\item \texttt{time} is the current model time, in hours;
\item \texttt{temp} is the daily mean $2\,\mathrm{m}$ global air temperature field, in
  $^{\circ}\mathrm{C}$;
\item \texttt{precip} is the global daily accumulated precipitation field,
  in $\mathrm{mm}$ (water equivalent, making no distinction
  between rain, snow, etc.);
\item \texttt{zonwind} and \texttt{merwind} are the daily mean global zonal and
  meridional components of the $10\,\mathrm{m}$ wind field, in
  $\mathrm{ms}^{-1}$;
\item \texttt{orog} is the global orography field, in $\mathrm{m}$.
\end{itemize}
%
For the positive degree-day mass-balance routine, which is currently the only
mass-balance model included with GLINT, the daily quantities given above are
necessary, and, as such, GLINT should be called once per day. With the
energy-balance mass-balance model currently being developed in the RAPID
project, hourly calls will be necessary. 
%
\subsubsection{Finishing off}
%
After the desired number of time-steps have been run, GLINT may have some
tidying up to do. To accomplish this, the subroutine \texttt{end\_glint}
must be called:
%
\begin{verbatim}
  call end_glint(ice_sheet)
\end{verbatim}
%
\subsubsection{API}
%
A detailed description of the GLINT API may be found in the appendices.
%
\subsubsection{Configuration}
%
GLINT uses the same configuration file format as the rest of GLIMMER. In the
case where only one GLIDE instance is used, all the configuration data for
GLINT and GLIDE can reside in the same file. Where two or more instances are
used, a top-level file specifies the number of model instances and the name of
a configuration file for each one. Possible configuration sections specific to
GLINT are as follows:
\begin{center}
  \tablefirsthead{%
    \hline
  }
  \tablehead{%
    \hline
    \multicolumn{2}{|p{0.98\textwidth}|}{\emph{\small continued from previous page}}\\
    \hline
  }
  \tabletail{%
    \hline
    \multicolumn{2}{|r|}{\emph{\small continued on next page}}\\
    \hline}
  \tablelasttail{\hline}
  \begin{supertabular}{|l|p{11cm}|}
%%%% 
    \hline
    \multicolumn{2}{|l|}{\texttt{[GLINT]}}\\
    \hline
    \multicolumn{2}{|p{0.98\textwidth}|}{Section specifying number of instances.}\\
    \hline
    \texttt{n\_instance} & (integer) Number of instances (default=1)\\
    \hline
%%%% 
    \hline
    \multicolumn{2}{|l|}{\texttt{[GLINT instance]}}\\
    \hline
    \multicolumn{2}{|p{0.98\textwidth}|}{Specifies the name of an
    instance-specific configuration file. Unnecessary if we only have one
    instance whose configuration data is in the main config file.}\\
    \hline
    \texttt{name} & Name of instance-sepcific config file (required).\\
    \hline
%%%% 
    \hline
    \multicolumn{2}{|l|}{\texttt{[GLINT projection]}}\\
    \hline
    \multicolumn{2}{|p{0.98\textwidth}|}{Projection info}\\
    \hline
    \texttt{projection} & {\raggedright (integer) type of projection:\\
       \begin{tabular}{lp{10cm}}
         1 & Lambert Equal Area \\
         2 & Spherical polar \\
         3 & Spherical stereographic (oblique)\\
         4 & Spherical stereographic (equatorial)\\
       \end{tabular}}\\
    \texttt{lonc} & Longitide of projection centre (degrees east)\\
    \texttt{latc} & Latitude of projection centre (degrees north)\\
    \texttt{cpx} & Local $x$-coordinate of projection centre (grid-points)\\
    \texttt{cpy} & Local $y$-coordinate of projection centre (grid-points)\\
    \texttt{std parallel} & Standard parallel of projection (degrees
    north). Only relevant when \texttt{projection}=3 (default=90.0)\\
    \texttt{earth\_radius} & Radius of the Earth (m) (default=$6.37\times10^6$)\\
    \hline
%%%% 
    \hline
    \multicolumn{2}{|l|}{\texttt{[GLINT climate]}}\\
    \hline
    \multicolumn{2}{|p{0.98\textwidth}|}{GLINT climate configuration}\\
    \hline
    \texttt{precip\_mode} & {\raggedright
      Method of precipitation downscaling: \\
      \begin{tabular}{lp{10cm}}
        {\bf 1} & Use large-scale precipitation rate\\
        2 & Use parameterization of \emph{Roe and Lindzen}\\
      \end{tabular}}\\
    \texttt{acab\_mode} & {\raggedright
      Mass-balance model to use:\\
      \begin{tabular}{lp{7cm}}
        {\bf 1} & Annual PDD mass-balance model (see section \ref{ug.mbal.pdd_scheme}) \\
        2 & Annual accumulation only\\
        3 & Hourly energy-balance model (RAPID - not yet available) \\
        4 & Daily PDD mass-balance model (no docs yet)\\
      \end{tabular}}\\
    \texttt{ice\_albedo} & Albedo of ice --- used for coupling to climate
    model (default=0.4) \\
    \texttt{lapse\_rate} & Atmospheric temperature lapse-rate, used to correct
    the atmospheric temperature onto the ice model orography. This should be
    \emph{positive} for temperature falling with height
    ($\mathrm{K}\,\mathrm{km}^{-1}$) (default=8.0) \\
    \hline
  \end{supertabular}
\end{center}
