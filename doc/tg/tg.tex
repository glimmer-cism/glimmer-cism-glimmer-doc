There are many tools for visualising netCDF files. In particular \href{http://meteora.ucsd.edu/~pierce/ncview\_home\_page.html}{\texttt{ncview}} and \href{http://www.epic.noaa.gov/java/ncBrowse/}{\texttt{ncBrowse}} are usefule viewers. We also provide a number of Python programs and classes using \href{http://gmt.soest.hawaii.edu/}{GMT} which produce postscript images of the GLIMMER netCDF files.

This chapter provides a brief manual and a cookbook of how to use the PyCF tools. 

\section{Getting and Installing Glimmer}
Glimmer is a relatively complex system of libraries and programs which
build on other libraries. This section documents how to get Glimmer
and its prerequisites, compile and install it. Please report problems
and bugs to the
\href{http://forge.nesc.ac.uk/mailman/listinfo/glimmer-discuss}{Glimmer
  mailing list}.
%
\subsection{Prerequisites}
\label{ug.sec.prereqs}
Glimmer is distributed as source code and needs a UNIX-like build
environment to be compiled in. On genuine UNIX systems
\href{http://www.gnu.org/software/make/}{GNU make} is suggested, since
the Makefiles may rely on some GNU make specific features. There are
two ways of getting the source code:
%
\begin{enumerate}
\item download a {\it released} version from the
  \href{http://glimmer.forge.nesc.ac.uk}{Glimmer
    website}\footnote{\texttt{http://glimmer.forge.nesc.ac.uk}}, or 
\item download the latest developers' version of Glimmer and friends
  from \href{http://forge.nesc.ac.uk/}{NeSCForge} using
  \href{http://www.gnu.org/software/cvs/}{CVS}. 
\end{enumerate}
%
For beginners, the latest release is recommended. More experienced
users may want to try the CVS version, as it will have all the latest
bug-fixes and new features. Section \ref{intro-versions} above
explains the meaning of Glimmer version numbers.

Whether you decide to build from a release or from CVS, there are two
essential prerequisites:
%
\begin{itemize}
\item {\bf A good f95 compiler.} Glimmer is known to
work with the NAGware f95, Intel \texttt{ifort} and later versions of GNU
\texttt{gfortran} compilers. Glimmer does not compile with the SUN WS 7.0 f95
compiler due to a compiler bug. The current SUN f95 compiler might
work, but has not been tested yet. 
\item {\bf The netCDF library.} Glimmer uses
  \href{http://www.unidata.ucar.edu/packages/netcdf/index.html}{netCDF}
  for data I/O. You will most likely need to compile and install the
  netCDF library yourself, since the binary packages usually do not
  contain the Fortran 90 bindings which are used by Glimmer. 
\end{itemize}
%
Additional packages are required if you want to build Glimmer from
CVS:
%
\begin{itemize}
\item{{\bf GNU autoconf} (2.54 or later)}
\item{{\bf GNU automake} (1.9.6 or later)}
\item{{\bf Python} (2.3 or later)}
\end{itemize}
%
Autoconf and automake are used to generate the build system,
while \href{http://www.python.org}{Python} is used for
analysing dependencies and for automatically generating parts of the
code.
%
\subsection{The Glimmer Directory Structure}\label{ug.sec.dirs}
The following commands describe the setup if you use the \texttt{bash}
shell. The setup works similarly for other shells. We suggest that you
install Glimmer in its own directory,
e.g. \texttt{/home/user/glimmer}. Assign the shell variable
\texttt{\$GLIMMER\_PREFIX} to this directory, i.e. \texttt{export
  GLIMMER\_PREFIX=/home/user/glimmer}. This directory will contain the
following sub-directories: 
\begin{center}
 \begin{tabular}{lp{9.5cm}}
   \texttt{\$GLIMMER\_PREFIX/bin} & executables are installed in this
   directory. Set your path to include this directory,
   i.e. \texttt{export PATH=\$PATH:\$GLIMMER\_PREFIX/bin}.  \\ 
   \texttt{\$GLIMMER\_PREFIX/include} & include and f95 module files
   will be installed in this directory. If you want to compile your
   own climate drivers set the compiler search path to include this
   directory. \\ 
   \texttt{\$GLIMMER\_PREFIX/lib} & the libraries get installed
   here. Set your linker to look in this directory for the Glimmer
   libraries if you want to compile your own climate drivers. \\ 
   \texttt{\$GLIMMER\_PREFIX/share} & data files get installed here. \\
   \texttt{\$GLIMMER\_PREFIX/src} & this is the only directory you
   need to create yourself. Unpack the Glimmer sources here.
 \end{tabular}
\end{center}
%
\subsection{Installing a Released Version of Glimmer}\label{ug.sec.tarball}
Download the Glimmer tarball from the Glimmer site and unpack it in
the \texttt{\$GLIMMER\_PREFIX/src} directory using 
\begin{verbatim}
tar -xvzf glimmer-VERS.tar.gz
\end{verbatim}
where \texttt{VERS} is the package version.

The package is then compiled using the usual GNU sequence of commands:
\begin{verbatim}
./configure --prefix=$GLIMMER_PREFIX [other_options]
make
make install
\end{verbatim} %$
%
The options and relevant environment variables are described in Table
\ref{ug.tab.env}. Environment variables can be added to the end of the
\texttt{configure} command:
%
\begin{verbatim}
./configure --prefix=$GLIMMER_PREFIX --with-netcdf=/usr/local/netcdf \
      FC=ifort F77=ifort
\end{verbatim} %$
%
\begin{table}[htbp]
  \centering
  \begin{tabular}{|l|p{8cm}|}
    \hline
    Variable & Description \\
    \hline
    \texttt{FC} & f95 compiler to be used (essential)\\
    \texttt{F77} & f77 compiler to be used (essential) \\
    \texttt{FCFLAGS} & flags passed to the f95 compiler \\
    \texttt{LDFLAGS} & linker flags\\
    \hline
    Option  & Description \\
    \hline
    \texttt{--help} & print help \\
    \texttt{--prefix=}{\it prefix} & the installation prefix,
    e.g. \texttt{GLIMMER} \\ 
    \texttt{--with-netcdf=}{\it location} & prefix where the netCDF
    library is installed \\ 
    \texttt{--with-blas=}{\it location} & extra libraries used to
    provide BLAS functionality. A built--in, non--optimised version of
    BLAS is used if this option is not used. \\ 
    \texttt{--with-enmabal=}{\it location} & location of
    energy-balance mass-balance model libraries and include files. \\ 
    \texttt{--enable-doc} & build documentation.\\
    \texttt{--enable-profile} & enable profiling of Glimmer (see
    Sec.~\ref{ug.sec.profile})\\ 
    \texttt{--enable-restarts} & enable full restarts (see
    Sec.~\ref{ug.sec.restarts})\\ 
    \texttt{--enable-sp} & enable single-precision compilation of
    model and interface. \\
    \hline
  \end{tabular}
  \caption{Environment variables and \texttt{configure} options used
  by Glimmer.} 
  \label{ug.tab.env}
\end{table}
%
\subsection{Installing from CVS}
Revisions of Glimmer are managed using CVS. You can download the
latest development version of Glimmer using the following sequence of
CVS commands: 
\begin{verbatim}
cvs -d:pserver:anonymous@forge.nesc.ac.uk:/cvsroot/glimmer login
cvs -z3 -d:pserver:anonymous@forge.nesc.ac.uk:/cvsroot/glimmer co glimmer
\end{verbatim} %$
When prompted for a password after the first command, simply press
{\it Enter}.

The CVS version does not include some automatically generated
files, which is why you will need GNU autotools and Python to build
from CVS (see section \ref{ug.sec.prereqs} above). The build scripts are
generated by running
%
\begin{verbatim}
./bootstrap
\end{verbatim}
%
in the \texttt{\$GLIMMER\_PREFIX/src} directory. The package is then
configured and built as described in Section \ref{ug.sec.tarball}.
%
\subsection{Profiling}\label{ug.sec.profile}
If you run the \texttt{configure} script with the option
\texttt{--enable-profile} you enable profiling of the model. By
default times are integrated over 100 time steps. You can change this
behaviour by setting the variable \texttt{PROFILE\_PERIOD}. The timing
data is written to the file \texttt{glide.profile} which contains 5
columns of data (see Table \ref{ug.tab.profile_format}).
\begin{table}[htbp]
  \centering
  \begin{tabular}{|l|l|}
    \hline
    Column 1 &total CPU time elapsed when data is written to file\\
    Column 2 &accumulated time spent on this block of calculations\\
    Column 3 &integer ID used to identify this block of calculations\\
    Column 4 &model year\\
    Column 5 &description of this block of calculations\\
    \hline
  \end{tabular}
  \caption{File format of profile data file.}
  \label{ug.tab.profile_format}
\end{table}
A Python script using the PyGMT library to visualise the profile is provided.
%
\subsection{Restarts}\label{ug.sec.restarts}
%
Glimmer provides {\it two} mechanisms for initialising the state of
the model from results of a previous run, written to a file:
%
\begin{itemize}
\item {\bf Hotstarts:} This is the simpler of the two mechanisms. A
  NetCDF file containing \emph{hotstart} data may be written as part
  of the regular output from the model, along with other output
  files. The variables written to the hotstart file are limited to
  those describing the state of the ice sheet, such as thickness,
  temperature distribution, etc --- only those that are necessary to
  initialise the model cleanly. The model may be initialised from any
  of the time-slices in the hotstart file during the usual
  initialisation sequence. For most applications, hotstarting should
  be entirely adequate.
\item {\bf Restarts:} Sometimes, however, it is desirable to be able
  to write the entire state of the model, including all temporary
  variables, accumulation arrays, etc, to file. The \emph{restart}
  mechanism enables this to be done. The difference in this case is
  that the model doesn't need to be initialised in the usual way --- its
  state is simply read from file, and the run continues as before. The
  implementation of this is complex, as it involves writing all
  elements of each derived type and all its sub-types to
  file. Consequently, full restarts are disabled by default at
  build-time, and must be enabled with the \texttt{--enable-restarts}
  option when the build is configured.
\end{itemize}
%
A full description of both methods are provided later in this manual.


\section{General Remarks}
All PyCF tools share some of the options. You can select a variable to be processed with the \texttt{-v} switch. A time slice is selected using the \texttt{-t} or \texttt{-T} switch. The difference between the two switches is that you select a time (in ka) with the \texttt{-t} switch and a time slice (a number starting at 0) with the \texttt{-T} switch. You can also use negative numbers for the \texttt{-T} switch to start counting at the end, e.g. \texttt{-T-1} selects the last time slice.

For some plots you can select more than one variable or time or use a number of input files. You can change the width of a plot with the \texttt{--width} switch. This is particularly useful if you are plotting more than one variable/time/file with \texttt{plotCFvar.py}.

All PyCF tools produce postscript files. You can send these files straight to a postscript printer. If you want to include images in a {\LaTeX} document you need to convert them to an encapsulated postscript file. \texttt{ps2epsi} is such a utility and is used for the examples shown here. You can also use \texttt{convert} which is part of \href{http://www.imagemagick.org/script/index.php}{ImageMagick} to convert the resulting postscript files to any other graphics format such as \texttt{PNG}.

You can always use the switch \texttt{-h} or \texttt{--help} to get some documentation of the switches. Some of the options have no effect or do not behave as you would expect. Please expirment with the tools and report bugs and make suggestions for improvement. 

\subsection{Using \texttt{plotCFvar.py}}
\texttt{plotCFvar.py} is used to plot a time slice of a 2D variable, such as ice thickness or basal velocities. Furthermore, a vertical slice can be specified if the field is three--dimensional such as the temperature or full velocity field. The displayed data can be clipped to the ice extent or area above sea--level. A three--dimensional effect can be created by illuminating the result.

PyCF comes with a set of colourmaps for some variables, like ice thickness, topography, temperatures, etc. With the \texttt{--colourmap} option you can specify a custom colourmap. You can use any GMT colour map. You can automatically calculate a rainbow colourmap with the option \texttt{--colourmap=None}.

\begin{pycf}{plotCFvar.py -vtopg fenscan.nc topo.ps}{\dir/figures/topo.eps}
plots the bedrock topgraphy. The \texttt{-v} switch selects the variable. The first time slice is processed if no time slice is specified on the command line. 

The GMT coastline database is used to add present--day coastlines. The colourmap is selected for a set of predefined colourmaps.
\end{pycf}

\begin{pycf}{plotCFvar.py -vis -cthk -iis -t -17 --land fenscan.nc is.ps}{\dir/figures/is.eps}
plots the ice surface which is calculated from the variables \texttt{topg} and \texttt{thk}. The switch \texttt{-t} select a time slice at -17ka. The data is clipped to the area covered by ice with the \texttt{-c} switch. \texttt{-i} adds illumination. The \texttt{--land} switch colours areas above sea--level in grey.
\end{pycf}

\begin{pycf}{plotCFvar.py -vbtemp -cthk -gbvel --pmt -t -17 --land fenscan.nc btemp.ps}{\dir/figures/btemp.eps}
plots the basal temperatures adjusted for its dependance on pressure (\texttt{--pmt}) and adds glyphs for the basal velocity directions with the \texttt{-g} switch. The density of glyphs is automatically adjusted to the size of the plot. Glyphs only indicate direction. If you want magnitudes as well you should just plot the basal velocity field with \texttt{-vbvel}.
\end{pycf}

\begin{pycf}{plotCFvar.py -vbvel --urg=20 60 -t -17 fenscan.nc bvel.ps}{\dir/figures/bvel.eps}
plots an enlargement of the basal velocity field. You can specify the geographic coordinates of the lower--left corner (\texttt{--llg}) and/or upper--right corner (\texttt{--urg}) to define the region to be plotted. Alternatively, you can also specify the corners in the projected coordinate system using \texttt{--llx} and \texttt{--urx}.
\end{pycf}

\begin{pycf}{plotCFvar.py -vbvel -cthk -t -17 --land -pprof --not\_p fenscan.nc bvel.ps}{\dir/figures/bvelp.eps}
plots a 2D field with a profile line. The \texttt{-p} selects a file containing coordinates of the profile. This option can be repeated to have more than one line. The \texttt{--not\_p} is short for \texttt{--not\_projected} and indicates that the coordinates have to be projected to the map coordinate system. Profile lines are labeled automatically, starting with A.
\end{pycf}

\subsection{Using \texttt{plot\_extent.py}}
\texttt{plot\_extent.py} plots the ice extent at a given time. This program is particularly useful to compare different runs. \texttt{plot\_extent.py} can also plot profile lines using \texttt{-p} switch like \texttt{plotCFvar.py}.

\begin{pycf}{plot\_extent.py -t -17 fenscan.nc fenscan-gthf.nc extent.ps}{\dir/figures/extent.eps}
plots the ice extent of simulations stored in \texttt{fenscan.nc} and \texttt{fenscan-gthf.nc} at time -17ka.
\end{pycf}

\section{Using \texttt{plotProfile.py}}
\texttt{plotProfile.py} can be used to plot variables along a profile line. Again, time slices can be selected and multiple variables plotted. A number of files can be plotted if the variable is not three--dimensional. If the variable is three--dimensional, like temperature, it is transformed from the $\sigma$-- to the $z$--coordinate system (see Chapter \ref{num.sec.sigma}).

\begin{pycf}{plotProfile.py -vis  -t -17 -p prof --not\_p fenscan.nc fenscan-gthf.nc is.ps}{\dir/figures/prof_is.eps}
  plots a profile of the ice elevation and bed rock topograpy at time -17ka.
\end{pycf}

\begin{pycf}{plotProfile.py -vis -vbvel -vbmelt -t -17 -p prof --not\_p $\backslash$ \newline   fenscan.nc fenscan-gthf.nc vars.ps}{\dir/figures/prof_vars.eps}
  stacks a number of variables on top of each other.
\end{pycf}

\begin{pycf}{plotProfile.py -vtemp  -t -18 -p prof --not\_p fenscan.nc temp.ps}{\dir/figures/prof_temp.eps}
  plots a profile of the internal temperature structure at -18ka.
\end{pycf}

