
\subsection{EISMINT: using \texttt{glimmer-example}}
As you might already know, the heart of GLIMMER is the actual ice sheet model
GLIDE. This is where ice physics are resolved etc. To model an ice sheet using
GLIDE, you at least need to provide it with information about the ice sheet
mass balance. To get you started with a real simple example climate driver,
download \texttt{glimmer-example} from the project homepage or via CVS, cd into
the directory and type
\begin{verbatim}
glide_launch.py example.config
\end{verbatim}
this will kick off a simple EISMINT-1 moving margin type model run. The results
are written to \texttt{example.nc}, use a viewer like ncview to visualise them.
Take a look at the \texttt{example.config} file and read the documentation on
the EISMINT type climate driver to better understand what is happening:\\

\begin{verbatim}
# configuration for the EISMINT-1 test-case # moving margin

[grid]
# grid sizes
ewn = 31
nsn = 31
upn = 11
dew = 50000
dns = 50000

[options]
temperature = 1
flow_law = 2
isostasy = 0
sliding_law = 4
marine_margin = 2
stress_calc = 2
evolution = 2
basal_water = 2
vertical_integration = 0

[time]
tend = 200000.
dt = 10.
ntem = 1.
nvel = 1.
niso = 1.

[parameters]
flow_factor = 1
geothermal = -42e-3

[CF default]
title: EISMINT-1 moving margin

[CF output]
name: example.nc
frequency: 1000
variables: thk uflx vflx bmlt temp
uvel vvel wvel
\end{verbatim}\\

The \texttt{[grid]} section sets up the topography for the model run. As this
is an EISMINT testcase, there is no real topography, ice is building up on a
flat surface, and the mass balance is parameterised as a function of distance
from the grid center. The grid used here is 31x31 cells large (\texttt{ewn x
nsn}), has 11 vertical layers (\texttt{upn=11}) and an internal cell spacing of
5000 (\texttt{dew} and \texttt{dns}).

The \texttt{[options]} sections determines the basic behaviour of the model:\\
\texttt{temperature = 1} resolves the temperature over the whole of the 11
layers of ice (instead of assuming ice to be isothermal), \texttt{isostasy = 0}
turns off the isostasy component, etc. (check the documentation).

In the \texttt{[time]} section, the end time of the model run is set to 200000
with a timestep size of 10 and keeping all internal update processes
(temperature and velocity) in line with the timesteps by setting their
multiplier to 1.

Flow factor and geothermal heat flux parameters are set in the
\texttt{[parameters]} section.

Finally, in the \texttt{[CF output]} section the results file name is given,
the \texttt{variables} that should be dumped to the file and the
\texttt{frequency} with which they are written to it (every 1000 years).

You might want to try and change some of the parameters, e.g. the flow factor
and rerun the model to see what happens. This is nice and straight forward, but
you might want to see a bit more of what GLIMMER can do.
